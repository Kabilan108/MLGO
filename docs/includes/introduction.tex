% introduction.tex
% Author: Tony Kabilan Okeke
% Date: 2023-03-23

\section{Introduction}

Differential expression analysis (DEA) is a commonly used technique to identify genes that
are differentially expressed between two or more biological conditions. Gene Ontology
(GO) enrichment analysis, on the other hand, is used to interpret the biological
functions and pathways associated with the differentially expressed genes. Both DEA and
GO enrichment analysis is commonly used in the analysis of RNA-seq data. However, both
can be affected by various sources of noise and bias, such as batch effects, sample
heterogeneity, and gene annotation errors \cite{koch2018beginner}. Thus, it is crucial to
develop robust and accurate methods to integrate DEA and GO enrichment analysis in a
data-driven manner.

\vspace{0.2cm}

To address this issue, we are developing a neural network model to predict the enriched
GO terms from the log fold change (logFC) values computed by DEA. Neural networks are
powerful machine learning models that can learn complex patterns and relationships from
high-dimensional data. In this project, we aim to develop an autoencoder model that can
learn a compressed latent representation of the logFC values. The autoencoder is composed
of an encoder that maps the input logFC values to a lower-dimensional latent space and a
decoder that reconstructs the input logFC values from the latent space. We hypothesize
that the latent space of the autoencoder will capture the essential features and patterns
of the logFC values. In our future work, we will use the latent space to predict the
enriched GO terms from the logFC values.

\vspace{0.2cm}

There have been several previous studies that have used neural networks to predict
gene expression or gene function from transcriptomic data. One such study is D-GEX,
which proposed a deep neural network model to predict gene expression levels from a set
of landmark genes \cite{D-GEX}. The D-GEX model was shown to accurately predict the
expression levels of thousands of genes with lower error rates that the current standard
(linear regression) \cite{D-GEX}. Our work differs from D-GEX in that we
focus on the integration of DEA and GO enrichment analysis. Additionally, by first
learning a latent representation of the DEA results, we will also be able to reduce
noise and bias in the data, improving the accuracy of the downstream predictions.
